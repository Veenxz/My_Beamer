%%%%%%%%%%%%%%%%%%%%%%%%%%%%%%%%%%%%%%%%%%%%%%%%%%%%%%%%%%%%%%%%%%%%%%%%%%%%%
%																			%
%							  My LaTeX slide             					%
%				     	Author:Veenxz Version:1.0        					%
%					   Beijing Forestry University							%
%						  Update Date:2020/06/09        					%
%						  													%
%%%%%%%%%%%%%%%%%%%%%%%%%%%%%%%%%%%%%%%%%%%%%%%%%%%%%%%%%%%%%%%%%%%%%%%%%%%%%


\documentclass[aspectratio=169]{beamer}	% Compile at least twice!
%\setbeamertemplate{navigation symbols}{}
\usetheme{Warsaw}
%\useinnertheme{rectangles}
%\useoutertheme{infolines}
\useoutertheme[title,section,subsection=true]{smoothbars}
 
% -------------------
% Packages
% -------------------
\usepackage{	
	amsmath,			% Math Environments
	amssymb,			% Extended Symbols
	enumerate,		    % Enumerate Environments
	graphicx,			% Include Images
	lastpage,			% Reference Lastpage
	multicol,			% Use Multi-columns
	multirow,			% Use Multi-rows
	pifont,			    % For Checkmarks
	stmaryrd			% For brackets
}
\usepackage[english]{babel}

\iffalse
\DeclareOptionBeamer{aspectratio}[169]{%
  \ifnum#1=1610% 16:10
    \beamer@paperwidth 16.00cm%
    \beamer@paperheight 10.00cm%
  \else\ifnum#1=169% 16:9
    \beamer@paperwidth 16.00cm%
    \beamer@paperheight 9.00cm%
  \else\ifnum#1=149% 14:9
    \beamer@paperwidth 14.00cm%
    \beamer@paperheight 9.00cm%
  \else\ifnum#1=54% 5:4
    \beamer@paperwidth 12.50cm%
    \beamer@paperheight 10.00cm%
  \else\ifnum#1=43% 4:3
    \beamer@paperwidth 12.80cm%
    \beamer@paperheight 9.60cm%
  \else\ifnum#1=32% 3:2
    \beamer@paperwidth 13.50cm%
    \beamer@paperheight 9.00cm%
  \fi\fi\fi\fi\fi\fi%
}
\fi

% -------------------
% Colors
% -------------------
\definecolor{UniOrange}{RGB}{212,69,0}
\definecolor{UniGray}{RGB}{62,61,60}
\definecolor{UniForestGreen}{RGB}{0,120,0}
\definecolor{UniMediumBlue}{RGB}{0,0,150}
%\definecolor{UniRed}{HTML}{B31B1B}
%\definecolor{UniGray}{HTML}{222222}
\setbeamercolor{title}{fg=UniMediumBlue}
\setbeamercolor{frametitle}{fg=UniMediumBlue}
\setbeamercolor{structure}{fg=UniMediumBlue}
\setbeamercolor{section in head/foot}{bg=UniMediumBlue}
\setbeamercolor{author in head/foot}{bg=UniMediumBlue}
\setbeamercolor{date in head/foot}{fg=UniMediumBlue}
\setbeamercolor{structure}{fg=UniMediumBlue}
\setbeamercolor{local structure}{fg=black}
\beamersetuncovermixins{\opaqueness<1>{0}}{\opaqueness<2->{15}}


% -------------------
% Fonts & Layout
% -------------------
\useinnertheme{default}
\usefonttheme{serif}
\usepackage{palatino}
\setbeamerfont{title like}{shape=\scshape}
\setbeamerfont{frametitle}{shape=\scshape}
\setbeamertemplate{itemize items}[circle]

%添加页码  4:3 55em  16:9 78em
\addtobeamertemplate{navigation symbols}{}{\hspace{78 em}  \usebeamerfont{footlines}%
	   \insertframenumber / \inserttotalframenumber \hspace{0.5 em}}
% -------------------
% Commands
% -------------------

% Special Characters
\newcommand{\N}{\mathbb{N}}
\newcommand{\Z}{\mathbb{Z}}
\newcommand{\Q}{\mathbb{Q}}
\newcommand{\R}{\mathbb{R}}
\newcommand{\C}{\mathbb{C}}

% Math Operators
\DeclareMathOperator{\im}{im}
\DeclareMathOperator{\Span}{span}

% Special Commands
\newcommand{\pf}{\noindent\emph{Proof. }}
\newcommand{\ds}{\displaystyle}
\newcommand{\defeq}{\stackrel{\text{def}}{=}}
\newcommand{\ov}[1]{\overline{#1}}
\newcommand{\ma}[1]{\stackrel{#1}{\longrightarrow}}
\newcommand{\twomatrix}[4]{\begin{pmatrix} #1 & #2 \\ #3 & #4 \end{pmatrix}}


% -------------------
% Tikz & PGF
% -------------------
\usepackage{tikz}
\usepackage{tikz-cd}
\usetikzlibrary{
	calc,
	decorations.pathmorphing,
	matrix,arrows,
	positioning,
	shapes.geometric
}
\usepackage{pgfplots}
\pgfplotsset{compat=newest}


% -------------------
% Theorem Environments
% -------------------
\theoremstyle{plain}
\newtheorem{thm}{Theorem}[section]
\newtheorem{prop}{Proposition}[section]
\newtheorem{lem}{Lemma}[section]
\newtheorem{cor}{Corollary}[section]
\theoremstyle{definition}
\newtheorem{ex}{Example}[section]
\newtheorem{nex}{Non-Example}[section]
\newtheorem{dfn}{Definition}[section]
\newtheorem{met}{Method}[section]
\theoremstyle{remark}
\newtheorem{rem}{Remark}[section] 
\numberwithin{equation}{section}

% -------------------
% Logo
% -------------------
%Only Title
\titlegraphic{\includegraphics[height=0.12\textwidth]{logo_new.png}}

%All
%\logo{\includegraphics[height=0.2\textwidth]{logo.png}}


% -------------------
% Title Page
% -------------------
\title{\textcolor{white}{\href{https://github.com/Veenxz/My_Beamer}{My \LaTeX\ slide}}}
\subtitle{\textcolor{white}{Subheading}}  
\author{ \href {mailto:veenzhou@gmail.com}{Qi-Zhi Zhou}}
\date{\footnotesize \today} 


% -------------------
% Content
% -------------------
\begin{document}


% Title Page
\begin{frame}
\titlepage
\end{frame}



% Introduction
\section{Introduction}



% Introduction-Background
\begin{frame}
Introduction-Background. \dots
\end{frame}



% Definitions & Examples
\begin{frame}
Here is my definition\dots

\begin{dfn}[01]
DFN
\end{dfn}

\begin{ex}
EG\\
\emph{Example}.
\end{ex}
\end{frame}



% Methods
\section{Methods}



% Method 1
\begin{frame}
\begin{met}[1]
What we do.
\end{met}
\end{frame}



% Results
\section{Results}



% Theorem
\begin{frame}
\begin{thm}[D.]
For all $n$, we have $n^2= n \cdot n$.
\end{thm}

\pf With massive loss of generality, let $n=1$. Then we have
	\[
	1=1^2= 1 \cdot 1= 1
	\]
Therefore by overwhelming hope, it must always be true. \qed
\end{frame}


% Disscussion
\section{Disscussion}



% Disscussion
\begin{frame}
My Disscussion \dots
\end{frame}



% Corollary
\begin{frame}
Most algebra you need to be true is true.
\begin{cor}
For all $n,m \in \N$, $(n+m)^2= n^2 + m^2$.
\end{cor}
\end{frame}



% Applications
\section{Applications}



% Application 1
\begin{frame}
	\begin{enumerate}[1.]
	\item Bleach is mostly water. \pause
	\item We are mostly water. \pause
	\item Therefore, we are bleach.
	\end{enumerate} \vspace{0.5cm}

Now we pause for the big reveal\dots \pause \vspace{0.3cm}

	\begin{itemize}
	\item I am clearly a master of logic.
	\item Masters of logic get Ph.D's.
	\item I have earned this.
	\end{itemize}
\end{frame}




% Conclusion
\section{Conclusion}



% Conclusion
\begin{frame}
	\begin{enumerate}[1]
	\item \dots 
	\item \dots 
	\item \dots
	\end{enumerate} \vspace{0.5cm}
\end{frame}




% Appendix
\section{Appendix}



% Appendix
\begin{frame}
Finally! Some Math! \vspace{1cm}

Here is some Math: $\int_1^\alpha \dfrac{x^2}{\sin x^2} \;dx$ and $\sum i^2$. \vspace{1cm}

But you could make this Math big inline with `displaystyle': $\ds \int_1^\alpha \dfrac{x^2}{\sin x^2} \;dx$ and $\ds \sum i^2$. \vspace{1cm}

And even more Math:
	\[
	\oint \vec{\nabla} \times \vec{F} \;dV = \sum_{n=1}^\infty \ov{p} \twomatrix{a}{b}{c}{d}
	\]
\end{frame}


% Questions
\begin{frame}
\begin{center} {\bfseries \Large Questions?} \end{center}
\end{frame}


\end{document}